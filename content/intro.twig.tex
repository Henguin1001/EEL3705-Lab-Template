\section{Abstract}
\label{sec:Abstract}
Circuit Analysis tools are extremely useful in the field of electrical engineering.
Tools like the Digilent Analog Discovery board, the focus of this lab, allow for important
measurements to be made. The purpose of this lab is to become familiar with this tool. This
is done by using it to analyze circuits of voltage and current dividers, as well as a
wheatstone bridge circuit. In order to complete the experiments the missing resistor
values had to be determined. Given specifications such as percentage of power each resistor
should use their values were calculated. After completing the experiments it was found that
the chosen values fell within range of the specifications. It was also proven that
the circuits obeyed conservation of electrical power, Kirchhoff's Voltage Law, and
Kirchhoff's Current Law. Following the experimental procedure served as a valuable
exercise in learning more about these tools and therefore completes this lab's purpose.


\section{Introduction}
\label{sec:Introduction}

\subsection{Problem Statement}
\label{sub:Problem Statement}
The problem is that analyzing circuits can be difficult without a range of tools for the job.
This becomes especially important as circuits become more complex. Tools such as the
Digilent Analog Discovery Board can greatly simplify many of these problems.

\subsection{Purpose of Experiment}
\label{sub:Purpose of Experiment}
The purpose of this experiment is to become more familiar with the Digilent Analog Discovery Board
Specifically, to analyze circuits of voltage and current dividers, as well as a
wheatstone bridge circuit.

\subsection{Scope of Experiment}
\label{sub:Scope of Experiment}
This lab will focus on simple DC circuits that have only resistive elements. The lab
is divided into three parts. The first is analysis of a voltage divider, in which the values of
the resistors were calculated in order to generate the proper distribution of power.  The voltages
across each resistor were measured and used to prove conservation of electric power,
and Kirchhoff's Voltage Law. The second is the analysis of a current divider, in which the current
through each resistor was measured and used to prove conservation of electric power and Kirchhoff's
Current Law. The resistors were also selected to generate the specified distribution of power
The third part is the construction of a Wheatstone bridge circuit. Given the values for three
of the four resistors the fourth was calculated.
