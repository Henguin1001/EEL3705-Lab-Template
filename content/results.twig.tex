








\subsection{Setup}
\label{sub:Setup}
Resistors were provided, a total of 11. They were then documented by
their bands and values in Table~\ref{tab:resultResistors}.
{{ mac.placeTable(table1, "List of Resistors Used", "resultResistors")}}


\subsection{Part A: Verification of Ohm’s law}
\label{sub:Part A}
{{ table.ohm(table2, "1", 1)}}
{{ table.ohm(table3, "2.2", 2)}}
\begin{enumerate}
  \item The unknown resistors are made of linear material which can be seen from their graphs in Figures~\ref{fig:plot1} and \ref{fig:plot2} .
  The graph indicates linearity by the fact that the trend line matches the
  samples points without diverging or deviating.
  {{ mac.image("img/plot1.jpg", "Plot of Verification of Ohm's Law using resistor $R_1$", "0.65", "plot1")}}
  {{ mac.image("img/plot2.jpg", "Plot of Verification of Ohm's Law using resistor $R_2$", "0.65", "plot2")}}
  \item $R_1$ and $R_2$ can be calculated by the slope of the IV graph.
  From the slope the resistances were calculated to be 1k, and 2.2k which is the
  the same as their resistance rating. Despite high accuracy there is low precision as
  the linear regression was done with a simple best fit line.
  \begin{gather}
    m = \frac{y_2-y_1}{x_2-x_1} = \frac{1.0 V - 0.0 V}{1.0 mA - 0.0 mA} = 1000 \frac{V}{A} = 1 k \Omega \\
    m = \frac{y_2-y_1}{x_2-x_1} = \frac{1.1 V - 0.0 V}{0.5 mA - 0.0 mA} = 2200 \frac{V}{A} = 2.2 k \Omega
  \end{gather}
\end{enumerate}

\vspace{1cm}

\subsection{Part B: Investigation of Electric Power Transfer }
\label{sub:Part B}
{{ table.power(table4, "5")}}
\begin{enumerate}
  \item \begin{itemize}
    \item As the load resistance increases the load voltage increases.
    There is a direct non-linear relationship between voltage and resistance. This can be seen by the blue line of plot
    figure \ref{fig:currentvoltage}.
    \item As the load resistance increases the load current decreases.
   There is a indirect non-linear relationship between current and resistance. This can be seen by the red line of the plot.
   \end{itemize}
  The graph indicates:
  \begin{equation}
    \text{ As } R_L \to \infty \hspace{1cm} \begin{matrix}
    V_L \to V_{in} = 5 V \\
    I_L \to 0
  \end{matrix}
  \end{equation}
  Testing various resistor values in the voltage divider equation yeilds the same
  values that were measured.
  \begin{gather}
    \text{For a voltage divider: } V_{out} = V_{in} \cdot \frac{R_L}{R_S+R_L} \\
    V_{out}(R_L \to 0) = (5 V) \cdot \frac{0}{(2.2 k \Omega)+0} = 0 \\
    V_{out}(R_L \to 6.8k ) = (5 V) \cdot \frac{6.8 k \Omega}{(2.2 k \Omega)+ 6.8 k \Omega} = 3.78V
  \end{gather}



  {{ mac.image("img/plotc.jpg", "Plot of Voltage and Current as a Function of Load Resistance $R_L$", "0.75", "currentvoltage")}}

  \item The load power increased as the load resistance increased till it reached a maximum.
  At this point the power decreased. It created a concave down curve represented in Figure~\ref{fig:loadpower}
  {{ mac.image("img/plotd.jpg", "Plot of Load Power $P_L$ as a Function of Load Resistance $R_L$", "0.7", "loadpower")}}

  \item The resistor that resulted in the maximum power transfer was the $2.0 k\Omega$
    resistor, but according to the regression of the plot (Figure~\ref{fig:loadpower}) the value would be $R_L = 2.7 k\Omega$
    Both these values fall close to the value of the source resistor which
    had a value of $R_s = 2.2 k\Omega$. To be consistent with the theorem for maximum
    power transfer the load resistor would be equal to the source resistance $R_S$.
  \item The resistor that resulted in the maximum efficiency was the $6.8k \Omega$
  resistor, but according to the regression of the plot (Figure~\ref{fig:loadeff}) the value would
  be $R_L = 75.6 k\Omega$. However, these points were the endpoints of the sample data
   and the plot respectively. In reality the maximum efficiency is only when the resistance
   reaches infinity. The end behavior of the graph indicates this by the horizontal asymptote at $100\%$.
   \[\text{As } R_L \to \infty , \eta \to 100\% \]
   {{ mac.image("img/plote.jpg", "Plot of Efficiency $\eta$ as a Function of Load Resistance $R_L$", "0.7", "loadeff")}}

\end{enumerate}
