






\subsection{Setup}
\label{sub:Setup}
Resistors were provided, a total of 8. They were then documented by
their bands and values in Table~\ref{tab:resultResistors}.
{{ mac.placeTable(table1, "List of Resistors Used", "resultResistors")}}

\subsection{Part A: Voltage divider design}
\label{sub:resultsA}
{{ table.divider(table2, "0.725", "Voltage") }}
\subsubsection{Post-lab Exercises}
\begin{enumerate}
  \item The Calculations were done in Equation \ref{eq:calc1} \begin{equation}
    \begin{aligned}
      \text{Given } & P_{R_3} = 0.5\times P_{eq} \\
      \text{Because $i$ is consistent}& \text{ there is a direct relationship} \\
      \text{ where }\hspace{0.5cm} &
      P_x = i^2 R_x \\
      \text{Therefore, } \hspace{0.5cm} & P_{R_3} = 0.5\times P_{eq}\\
      \Leftrightarrow \hspace{0.5cm} & R_3 = 0.5 \times R_{eq}\\
      \text{And substituting for } R_{eq} \hspace{0.5cm} & R_3 = 0.5 \times (2.5k + R_3) \\
      & R_{3} = 2.5 k\Omega
    \end{aligned}\label{eq:calc1}
  \end{equation}
  \item \begin{equation}
    \begin{aligned}
      V_{in} & = V_{1}+V_{2}+V_{3}\\
      & = 0.453 + 0.634 + 0.928\\
      & = 2.014 V\\ \\
      I_{in} & = 0.36 mA\\ \\
      R_{in} & = \frac{V}{I}\\
      & = \frac{2.014 V}{0.36 mA}\\
      & =  5.6k \Omega\\ \\
      P_{in} & = I_{in} V_{in}\\
      & = (0.36 mA)(2.014 V)\\
      & =  0.725 mW\\
    \end{aligned}
  \end{equation}
  \item The values are relatively close coming within 19.1\% of each other.
  This is reasonable value considering all the sources of error. Such as the
  Tolerance of the resistors.
  \begin{equation}
    \begin{aligned}
      R_{EQ} & = R_1+R_2+R_3 \\
      & = 1k + 1.5k + 2.2k \\
      & = 4.7 k \Omega \\
      Error &= 100\% \times \frac{R_{in} - R_{EQ}}{R_{EQ}}\\
      &=  100\% \times \frac{5.6 - 4.7}{4.7}\\
      &= 19.1\%\\
    \end{aligned}
  \end{equation}
  \item The results are consistent with the voltage divider rule. This can be
  seen by the fact that the lowest resistance value (1k) drops the least Voltage
  and the highest value (2.2k) has the greatest voltage drop. Also a voltage divider
  has consistent current through each resistor which is true from the results.
  \item The results are consistent with the conservation of electric power because
  the sum of power is equal to the power in.
  \begin{equation}
    \begin{aligned}
      P_{in} & = 0.725 mW\\
      P_{total} & = P_{R1}+P_{R2}+P_{R3} \\
      & = 0.163 mW + 0.228 mW + 0.334 mW \\
      & = 0.725 mW \\
    \end{aligned}
  \end{equation}
  \item The results are consistent with KVL. This can be seen by the fact that the
  voltage measured is $2.0 V$ and the sum of each voltage is $2.0 V$ as well.
\end{enumerate}

\subsection{Part B: Current divider design}
\label{sub:resultsB}
In order to determine the values to use, the calculations were made in Equation~\ref{eq:calc2}
\begin{equation}\label{eq:calc2}
 \begin{split}
   \text{Given } \hspace{0.5cm} & P_{R_3} = 0.5\times P_{total} \\
   \text{Implies } \hspace{0.5cm} & P_{R_1 \parallelsum R_2} = 0.5\times P_{total}\\
   \text{Therefore } \hspace{0.5cm} & R_1 \parallelsum R_2 = R_3 = 1.5 k\Omega \\
   R_{eq} & = R_1 \parallelsum R_2 \parallelsum R_3\\
   & = 1.5 k\Omega \parallelsum 1.5 k\Omega \\
   & = 0.75 k\Omega \\
   i_{eq} & = \frac{V_s}{0.75 k}  \\
   & = 1.333\times 10^{-3}  \cdot V_S \\
   R_1 & = \frac{V_S}{0.2 \cdot i_{eq}} \\
   & = \frac{V_S}{0.2 \cdot 1.333\times 10^{-3}  \cdot V_S} \\
   & = \frac{0.75 V}{0.2 mA} = 3.75 k\Omega \\
   R_2 & = \frac{0.75 V}{0.3 mA} = 2.5 k\Omega \\
   R_3 & = \frac{0.75 V}{0.5 mA} = 1.5 k\Omega \\
 \end{split}
\end{equation}

{{ table.divider(table3, "5.42", "Current") }}
\subsubsection{Post-lab Exercises}
\begin{enumerate}
  \item \begin{equation}
    \begin{aligned}
      V_{in} & = 2.0 V\\ \\
      I_{in} & = I_{1}+I_{2}+I_{3}\\
      & = 0.56 mA + 0.86 mA + 1.29 mA\\
      & = 2.71 mA\\ \\
      R_{in} & = \frac{V}{I}\\
      & = \frac{2.0 V}{2.71 mA}\\
      & =  740 \Omega\\ \\
      P_{in} & = I_{in} V_{in}\\
      & = (2.71 mA)(2.0 V)\\
      & =  5.42 mW\\
    \end{aligned}
  \end{equation}
  \item The values are close coming within 5.4\% of each other.
  This is good value considering all the sources of error.
  \begin{equation}
    \begin{aligned}
      R_{EQ} & = R_1 \parallelsum R_2 \parallelsum R_3 \\
      & = 1k \parallelsum 1.5k \parallelsum 2.2k \\
      & = 702 \Omega \\
      Error &= 100\% \times \frac{R_{in} - R_{EQ}}{R_{EQ}}\\
      &=  100\% \times \frac{740 - 702}{702}\\
      &= 5.4\%\\
    \end{aligned}
  \end{equation}
  \item The results are consistent with the current divider rule. This can be
  seen by the fact that the lowest resistance value (1.5) draws the greatest current
  and the highest value (3.3k) has the least current draw. Also a current divider
  has consistent voltage across each resistor which is true from the results.
  \item The results are consistent with the conservation of electric power because
  the sum of power is equal to the power in.
  \begin{equation}
    \begin{aligned}
      P_{in} & = 5.42 mW\\
      P_{total} & = P_{R1}+P_{R2}+P_{R3} \\
      & = 1.12 mW + 1.72 mW + 2.58 mW \\
      & = 5.42 mW \\
    \end{aligned}
  \end{equation}
  \item The results are consistent with KCL. This can be seen by the fact that the
  current measured is $2.7 mA$ and the sum of each voltage is $2.7 mA$ as well.
\end{enumerate}

\subsection{Part C: Wheatstone Bridge}
\label{sub:resultsC}
\subsubsection{Post-lab Exercises}
\begin{enumerate}
  \item Equation \ref{eq:calc3} shows how to solve for $R_x$ given $R_1, R_2, R_3$ \\
   \begin{equation}
    \begin{aligned}
      \frac{R_1}{R_2} & = \frac{R_3}{R_X}\\
      R_X & = \frac{R_2 R_3}{R_1}\\
      & = \frac{1.5k\Omega 2.2k\Omega}{1k \Omega}\\
      & = 3.3 k\Omega \\
    \end{aligned} \label{eq:calc3}
  \end{equation}
\end{enumerate}
