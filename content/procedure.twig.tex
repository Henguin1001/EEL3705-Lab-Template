

From the Lab Manual \cite{manual} :
\subsection{Part A: Voltage Divider Design}
\label{sub:Procedure Part A}
Build a voltage divider that satisfies the given power requirements using the circuit
given in Figure~\ref{fig:voltage}  Given, $R_1= 1 k\Omega$, $R_2 = 1.5 k\Omega$.
{{ mac.image("img/c1.png", "Voltage Divider Design", "0.65", "voltage")}}
\begin{enumerate}
  \item Determine the proper resistance for $R_3$ such that it absorbs 50\% of the power delivered by
  the voltage source $V_s$. \\
  \textbf{The calculations are done in equation \ref{eq:calc1}}
  \item Build the circuit shown in Figure~\ref{fig:voltage}  on the bread board.
  \item Make the following connections between the Analog Discovery pins and the nodes on the
  board:
  \begin{itemize}
    \item 1 – WaveGen1, W1 (Solid Yellow Line)
    \item 1 – Scope Channel, 1+ (Solid Orange)
    \item 0 – Ground (Black)
    \item 2 – Scope Channel, 1- (Striped Orange)
  \end{itemize}
  \item Activate the Digilent Waveforms button which would display a window (Digilent
  WaveForms 1) as shown in Figure~\ref{fig:awg}. The first step is to adjust the WaveGen.
  Click on the WaveGen button and set the arbitrary waveform generator to generate a DC
  voltage of 2V offset.
  {{ mac.image("img/awg.png", "Digilent Analog Discovery WaveGen setup", "0.65", "awg")}}
  \item After making the adjustments on wave generator click the Run AWG1
  \item To observe the output, activate the voltmeter by clicking on the Voltmeter under the
  More Instruments tab.
  \item Record the voltage across the resistor $R_1$ and fill in the table by making the relevant
  measurement across all the three resistors.
  \item The obtained measurements should be included in the lab report.
  \item Perform the pertinent calculations based on the measured data to complete the table
\end{enumerate}
\vspace{1cm}

\subsection{Part B: Current Divider Design}
\label{sub:Procedure Part B}
Build a current divider that satisfies the given power requirements with the circuit
configuration in Figure~\ref{fig:current} Given $R_3 = 1.5 k\Omega$
{{ mac.image("img/c2.png", "Current Divider Design", "0.65", "current")}}
\begin{enumerate}
  \item Determine the proper resistances for $R_1$ and $R_2$ such that they absorb 20\% and 30\% of
  the power delivered by the voltage source Vs, respectively.\\
  \textbf{The calculations are done in Equation \ref{eq:calc2}}
  \item Build the circuit shown in Figure~\ref{fig:current}  on the bread board.
  \item Make the following connections between the Analog Discovery pins and the nodes on the
  board:
  \begin{itemize}
    \item 1 – WaveGen1, W1 (Solid Yellow Line)
    \item 1 – Scope Channel, 1+ (Solid Orange)
    \item 0 – Scope Channel, 1- (Striped Orange)
    \item 0 – Ground (Black)
  \end{itemize}
  \item Activate the Digilent Waveforms button which would display a window (Digilent
  WaveForms 1) as shown in Figure~\ref{fig:awg} . The first step is to adjust the WaveGen. Click on
  the WaveGen button and set the arbitrary waveform generator to generate a DC voltage
  of 2 volts offset.
  \item After making the adjustments on wave generator click the Run AWG1
  \item To observe the output, activate the voltmeter by clicking on the Voltmeter under the
  More Instruments tab.
  \item Make the relevant measurements to fill in the voltage and current columns in Table 3.
  \item Make the necessary calculations to complete the rest of the table
\end{enumerate}

\subsection{Part C: Wheatstone Bridge}
\label{sub:Procedure Part C}
Build a Wheatstone bridge circuit to determine the unknown resistor $R_x$ with the circuit
configuration in Figure~\ref{fig:wheatstone} . Given $R_1 = 1 k\Omega$, $R_2 = 1.5 k\Omega$, $R_3 = 2.2 k\Omega$.
{{ mac.image("img/c3.png", "Wheatstone Design", "0.65", "wheatstone")}}
\begin{enumerate}
  \item Build the circuit shown in Figure~\ref{fig:wheatstone}  on the bread board.
  \item Make the following connections between the Analog Discovery pins and the nodes on the
  board:
  \begin{itemize}
    \item 1 – WaveGen1, W1 (Solid Yellow Line)
    \item 1 – Scope Channel, 1+ (Solid Orange)
    \item 0 – Scope Channel, 1- (Striped Orange)
    \item 0 – Ground (Black)
  \end{itemize}
  \item Activate the Digilent Waveforms button which would display a window (Digilent
  WaveForms 1) as shown in Figure~\ref{fig:awg} . The first step is to adjust the WaveGen. Click on
  the WaveGen button and set the arbitrary waveform generator to generate a DC voltage
  of 5V offset.
  \item After making the adjustments on wave generator click the Run AWG1
  \item To observe the output, activate the voltmeter by clicking on the Voltmeter under the
  More Instruments tab.
  \item Make the relevant measurements to find the value of the unknown resistor $R_X$.
  \item \textbf{The calculations are done in Equation \ref{eq:calc3}}

\end{enumerate}
