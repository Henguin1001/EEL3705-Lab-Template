

From the Lab Manual \cite{manual} :
\subsection{Part A: Verification of Ohm's Law}
\label{sub:Procedure Part A}
\begin{enumerate}
  \item Build the circuit shown in Figure~\ref{fig:ohmschem}  on the bread board.
  {{ mac.image("img/c1.png", "Schematic for Verifying Ohm's Law", "0.75", "ohmschem")}}
  \item Make the following connections between the Analog Discovery pins and the nodes on the
  board:
  \begin{itemize}
    \item 1 – WaveGen1, W1 (Solid Yellow Line)
    \item 1 – Scope Channel, 1+ (Solid Orange)
    \item 0 – Ground (Black)
    \item 0 – Scope Channel, 1- (Striped Orange)
  \end{itemize}

  \item Activate the Digilent Waveforms button which would display a window (Digilent
  WaveForms 1) as shown in Figure~\ref{fig:dadwg}. The first step is to adjust the WaveGen.
  Click on the WaveGen button and set the arbitrary waveform generator to generate a DC
  voltage of amplitude 0 volts and 0 offset.
  {{ mac.image("img/awg.png", "Digilent Analog Discovery WaveGen setup", "0.75", "dadwg")}}


  \item After making the adjustments on wave generator click the Run AWG1
  \item To observe the output, activate the voltmeter by clicking on the Voltmeter under the
  More Instruments tab.
  \item Record the voltage across the resistor and fill in the table as the voltage source varies from
  0 to 5 V. Replace resistor R1 with R2 and fill in Table 3 by varying the voltage source from 0 to 5
  V.
\end{enumerate}
\vspace{1cm}

\subsection{Part B: Investigation of Electric Power Transfer}
\label{sub:Procedure Part B}
\begin{enumerate}
  \item Build the circuit shown in Figure~\ref{fig:maxpowerschem}  on the bread board.
  {{ mac.image("img/c2.png", "Schematic for Measuring Maximum Power Transfer", "0.75", "maxpowerschem")}}
  \item Make the following connections between the Analog Discovery pins and the nodes on the
  board:
  \begin{itemize}
    \item 1 – WaveGen1, W1 (Solid Yellow Line)
    \item 2 – Scope Channel, 1+ (Solid Orange)
    \item 0 – Scope Channel, 1- (Striped Orange)
    \item 0 – Ground (Black)
  \end{itemize}
  \item Activate the Digilent Waveforms button which would display a window (Digilent
  WaveForms 1) as shown in Figure~\ref{fig:dadwg}. The first step is to adjust the WaveGen.
  Click on the WaveGen button and set the arbitrary waveform generator to generate a DC
  voltage of amplitude 5 volts and 0 offset.
  \item After making the adjustments on wave generator click the Run AWG1
  \item To observe the output, activate the voltmeter by clicking on the Voltmeter under the
  More Instruments tab.
  \item Record the voltage across the load resistor ($R_L$) and the current through the load resistor
  ($R_L$).
  \item Replace the first load resistor with the second load resistor and record the values in the
  table. Repeat this process till all the resistances are filled.
  \item Make the necessary calculations (Equation \ref{eq:1}) to complete the rest of the table and plot efficiency ($\eta$) as a
function of the load resistance.
  \begin{equation}
    \begin{gathered}
      P_L = V_L \times I_L \\
      P_{in} = V_S \times I_L \\
      \eta = 100 \times \frac{P_L}{P_{in}}
    \end{gathered}\label{eq:1}
  \end{equation}



\end{enumerate}
