\section{Discussion}
\label{sec:Discussion}
After completing the experiments it was found that the resistors
provided did indeed obey ohm's law. A simple plot proves this because
the current and voltage was linear which. This was expected because there
is no reason why ohm's law wouldn't apply to simple resistors. Originally
the instructions were misunderstood. The error was that the current through the
resistor did not need to be measured. Furthermore, the Digilent Analog
Discovery board could not measure current. Instead, the current was measured
using the ELVIS II board.

For the power transfer circuit it was found that the voltage across
the load resistor had a direct relationship to the resistance, and the current an indirect
relationship. The calculations showed that maximum power transfer happens when the
load resistance is equal to that of the source resistance, and when the efficiency is at 50\%.
The maximum efficiency would be when the circuit is left unconnected where the resistance is
at its largest. This was expected as well, because prior knowledge lead to the
hypothesis that the maximum is the same is that thevenin resistance. The prelab
exercise also involved simulating the same experimental which lead to the same
results.

\section{Conclusion and Recommendations}
\label{sec:Conclusion}
This experiment was a valuable introduction to Digilent Analog Discovery board
and served its purpose to become familiar with the platform. The lab reveal any
new knowledge about circuitry as the experiments yielded the same results as
hypothesized. In the future the lab could have been done better if done using
equipment that can measure current. This was a minor setback that required
introducing new equipment to take up this task. This lab was helpful and will
provide a sound step in the following labs that build off of these experiments.
